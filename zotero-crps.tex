
@article{van_der_geest_effects_2011,
	title = {The Effects of Employment on Longitudinal Trajectories of Offending: A Follow-up of High-Risk Youth from 18 to 32 Years of Age},
	volume = {49},
	issn = {00111384},
	url = {http://doi.wiley.com/10.1111/j.1745-9125.2011.00247.x},
	doi = {10.1111/j.1745-9125.2011.00247.x},
	shorttitle = {{THE} {EFFECTS} {OF} {EMPLOYMENT} {ON} {LONGITUDINAL} {TRAJECTORIES} {OF} {OFFENDING}},
	version = {18},
	pages = {1195--1234},
	number = {4},
	journaltitle = {Criminology},
	author = {Van Der Geest, Victor R. and Bijleveld, Catrien C. J. H. and Blokland, Arjan A. J.},
	urldate = {2018-11-22},
	date = {2011-11},
	langid = {english}
}

@article{siwach_unemployment_2018,
	title = {Unemployment shocks for individuals on the margin: Exploring recidivism effects},
	volume = {52},
	issn = {0927-5371},
	url = {http://www.sciencedirect.com/science/article/pii/S0927537116303578},
	doi = {10.1016/j.labeco.2018.02.001},
	shorttitle = {Unemployment shocks for individuals on the margin},
	abstract = {This paper analyzes the impact of unemployment on the likelihood of returning to criminal activity for a sample of individuals with criminal records who are actively seeking employment. I use administrative data from the New York State Division of Criminal Justice Services, Department of Labor, and Department of Health, to track unemployment and arrest outcomes for this sample between 2008 and 2014. To identify the unemployment–arrest relationship, I use industry-specific variation in unemployment trends caused by the recession in 2008–2009, along with individual fixed effects to control for time-constant individual heterogeneity. The 2SLS estimates suggest that increased unemployment has large effects on rearrests for individuals with criminal records who are now active in the labor market, with substantial heterogeneity by race and sex. The results suggest larger estimates than those typically found in literature, indicating that targeting employment programs at those “on the margin” could substantially reduce rearrest rates for such individuals.},
	version = {16},
	pages = {231--244},
	journaltitle = {Labour Economics},
	shortjournal = {Labour Economics},
	author = {Siwach, Garima},
	urldate = {2018-11-22},
	date = {2018-06-01},
	keywords = {Arrests, Crime, Healthcare industry, Recession, Recidivism, Unemployment}
}

@article{siwach_criminal_2017,
	title = {Criminal background checks and recidivism: Bounding the causal impact},
	volume = {52},
	issn = {0144-8188},
	url = {http://www.sciencedirect.com/science/article/pii/S0144818817300327},
	doi = {10.1016/j.irle.2017.08.002},
	shorttitle = {Criminal background checks and recidivism},
	abstract = {This paper estimates the effect of employment denial based on a criminal background check on recidivism outcomes for individuals with convictions who are provisionally hired in the New York State healthcare industry. Using institutional knowledge about the New York State Department of Health’s screening process, I build structural assumptions on potential outcomes for different subsamples in my data, which partially identifies the Average Treatment Effects. I find a 0–2.2 percentage-point increase in the likelihood of subsequent arrests caused by employment denial, with substantial heterogeneity across the sample. Specifically, I find that the a priori highest risk individuals are most likely to be impacted by a loss of employment opportunity based on their criminal background. Policy implications of these results are discussed.},
	version = {15},
	pages = {74--85},
	journaltitle = {International Review of Law and Economics},
	shortjournal = {International Review of Law and Economics},
	author = {Siwach, Garima},
	urldate = {2018-11-22},
	date = {2017-10-01},
	keywords = {Bounds, Criminal background checks, Employment, Recidivism}
}

@report{neal_prison_2014,
	title = {The Prison Boom and the Lack of Black Progress after Smith and Welch},
	url = {http://www.nber.org/papers/w20283},
	abstract = {More than two decades ago, Smith and Welch (1989) used the 1940 through 1980 census files to document important relative black progress. However, recent data indicate that this progress did not continue, at least among men. The growth of incarceration rates among black men in recent decades combined with the sharp drop in black employment rates during the Great Recession have left most black men in a position relative to white men that is really no better than the position they occupied only a few years after the Civil Rights Act of 1965. A move toward more punitive treatment of arrested offenders drove prison growth in recent decades, and this trend is evident among arrested offenders in every major crime category. Changes in the severity of corrections policies have had a much larger impact on black communities than white communities because arrest rates have historically been much greater for blacks than whites.},
	version = {14},
	number = {20283},
	institution = {National Bureau of Economic Research},
	type = {Working Paper},
	author = {Neal, Derek and Rick, Armin},
	urldate = {2018-11-22},
	date = {2014-07},
	doi = {10.3386/w20283}
}

@report{chien_second_2018,
	location = {Rochester, {NY}},
	title = {The Second Chance Gap},
	url = {https://papers.ssrn.com/abstract=3265335},
	abstract = {Over the last decade, dozens of states have enacted “second chance” reforms that increase the eligibility of individuals charged or convicted of crimes to, upon application, shorten or downgrade their past convictions, clean their criminal records, and/or regain the right to vote. While much fanfare has accompanied the increasing availability of “second chances,” less is known about their uptake. This study introduces the concept of the “second chance gap” - the gap between eligibility  and delivery of certain forms of second chance relief, and sizes it in connection with several initiatives. It documents "uptake gaps" in association with Presidential Clemency and Ca reclassification to be large, with 90\%   of the eligible population not receiving their second chance.  Applying laws from alll 50 states to a random sample of records associated with {\textasciitilde}15,000 adults seeking on-demand jobs in the past 20 months , I estimate that 40-50\% or more of Americans with criminal (court) records could clean their record on the basis of one or more state clearance criteria, for a conservative estimate of 25-30M of Americans who could, but have not, partially or fully cleared their criminal records under existing law. These findings suggest that in many cases, the majority of second chances have been missed chances, due to administrative factors like low awareness and high-cost, high-friction application processes and backlog. To close the second chance gaps and unlock opportunities for individuals with criminal histories, this Essay argues, policymakers should consider automating second chances, and in the process burden-shifting, centralizaing, and ensuing consistency in the implementation of second chances. Ensuring that the design and administration of second chance laws reflect their intent can help remove the red tape, not steel bars, that stand in the way of second chances.},
	version = {13},
	number = {{ID} 3265335},
	institution = {Social Science Research Network},
	type = {{SSRN} Scholarly Paper},
	author = {Chien, Colleen V.},
	urldate = {2018-11-22},
	date = {2018-09-15},
	langid = {english},
	keywords = {computational policy, cost-benefit analysis, criminal records, uptake analysis}
}

@report{u.s._census_bureau_american_2015,
	title = {American Community Survey},
	url = {https://www2.census.gov/programs-surveys/acs/methodology/questionnaires/2016/quest16.pdf?#},
	version = {12},
	number = {{ACS}-1({INFO})(2016)},
	type = {2016 Questionnaire},
	author = {{U.S. Census Bureau}},
	urldate = {2018-11-03},
	date = {2015}
}

@report{barrington_cornell_2017,
	title = {Cornell Project for Record Assistance Questionnaire - A-filers},
	rights = {Attribution-{ShareAlike} 4.0 International},
	url = {https://hdl.handle.net/1813/60021},
	abstract = {Questionnaire used by the Cornell Project for Records Assistance, part of the Cornell University {ILR} School (Industrial and Labor Relations), to collect information about respondents, their work history, their involvement with the criminal justice system, if any, and their choice of remedy under the settlement of the Gonzalez et al v. Pritzker case. The responses were used to inform support and remedy provided to respondents.},
	version = {6},
	number = {60021},
	institution = {Cornell University},
	type = {Questionnaire},
	author = {Barrington, Linda and Bigler, Esta R. and Cornwell, Erin York and Enayati, Hassan and Vilhuber, Lars and Wells, Martin T.},
	urldate = {2018-11-03},
	date = {2017-05},
	langid = {american}
}

@report{barrington_cornell_2017-1,
	title = {Cornell Project for Record Assistance Questionnaire - B-filers},
	rights = {Attribution-{ShareAlike} 4.0 International},
	url = {https://hdl.handle.net/1813/60022},
	abstract = {Questionnaire used by the Cornell Project for Records Assistance, part of the Cornell University {ILR} School (Industrial and Labor Relations), to collect information about respondents, their work history, their involvement with the criminal justice system, if any, and 
 their choice of remedy under the settlement of the Gonzalez et al v. Pritzker case. These respondents had chosen not to receive remedy, information was collected to understand their choice.},
	version = {4},
	number = {60022},
	institution = {Cornell University},
	type = {Questionnaire},
	author = {Barrington, Linda and Bigler, Esta R. and Cornwell, Erin York and Enayati, Hassan and Vilhuber, Lars and Wells, Martin T.},
	urldate = {2018-11-03},
	date = {2017-05},
	langid = {american}
}